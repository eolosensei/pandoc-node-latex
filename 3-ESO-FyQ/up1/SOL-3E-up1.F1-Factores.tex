% Options for packages loaded elsewhere
\PassOptionsToPackage{unicode}{hyperref}
\PassOptionsToPackage{hyphens}{url}
\PassOptionsToPackage{dvipsnames,svgnames*,x11names*}{xcolor}
%
\documentclass[
  spanish,
]{article}
\usepackage[sfdefault,lining]{FiraSans}
\usepackage{amsmath}
\usepackage{ifxetex,ifluatex}
\ifnum 0\ifxetex 1\fi\ifluatex 1\fi=0 % if pdftex
  \usepackage[T1]{fontenc}
  \usepackage[utf8]{inputenc}
  \usepackage{textcomp} % provide euro and other symbols
  \usepackage{amssymb}
\else % if luatex or xetex
  \usepackage{unicode-math}
  \defaultfontfeatures{Scale=MatchLowercase}
  \defaultfontfeatures[\rmfamily]{Ligatures=TeX,Scale=1}
\fi
% Use upquote if available, for straight quotes in verbatim environments
\IfFileExists{upquote.sty}{\usepackage{upquote}}{}
\IfFileExists{microtype.sty}{% use microtype if available
  \usepackage[]{microtype}
  \UseMicrotypeSet[protrusion]{basicmath} % disable protrusion for tt fonts
}{}
\makeatletter
\@ifundefined{KOMAClassName}{% if non-KOMA class
  \IfFileExists{parskip.sty}{%
    \usepackage{parskip}
  }{% else
    \setlength{\parindent}{0pt}
    \setlength{\parskip}{6pt plus 2pt minus 1pt}}
}{% if KOMA class
  \KOMAoptions{parskip=half}}
\makeatother
\usepackage{xcolor}
\IfFileExists{xurl.sty}{\usepackage{xurl}}{} % add URL line breaks if available
\IfFileExists{bookmark.sty}{\usepackage{bookmark}}{\usepackage{hyperref}}
\hypersetup{
  pdflang={es-ES},
  colorlinks=true,
  linkcolor=Maroon,
  filecolor=Maroon,
  citecolor=Blue,
  urlcolor=Blue,
  pdfcreator={LaTeX via pandoc}}
\urlstyle{same} % disable monospaced font for URLs
\usepackage[a4paper,headheight=10mm,headsep=10mm,top=30mm,right=25mm]{geometry}
\setlength{\emergencystretch}{3em} % prevent overfull lines
\providecommand{\tightlist}{%
  \setlength{\itemsep}{0pt}\setlength{\parskip}{0pt}}
\setcounter{secnumdepth}{-\maxdimen} % remove section numbering
\everymath{\displaystyle}
\ifxetex
  % Load polyglossia as late as possible: uses bidi with RTL langages (e.g. Hebrew, Arabic)
  \usepackage{polyglossia}
  \setmainlanguage[]{}
\else
  \usepackage[shorthands=off,main=spanish]{babel}
\fi
\ifluatex
  \usepackage{selnolig}  % disable illegal ligatures
\fi

\author{}
\date{}


% % MODIFICATIONS MADE BY ME

% MODIFICATIONS MADE BY ME

%\usepackage[sfdefault,lining]{FiraSans} %% option 'sfdefault' activates Fira Sans as the default text font
\usepackage[fakebold]{firamath-otf}
\renewcommand*\oldstylenums[1]{{\firaoldstyle #1}}
\setmathfont{Fira Math}


% SIUNITX PACKAGE %

% Cargar siunitx
\usepackage{siunitx}

% Configuración general
\sisetup{detect-all}

\sisetup{
    exponent-product        = \cdot,
    per-mode                = reciprocal,
    output-decimal-marker   = {,},
    group-digits            = integer,
    %text-celsius            = ^^b0\kern -\scriptspace C, % soluciona problemas con el símbolo de grados
    %math-celsius            = ^^b0\kern -\scriptspace C,
    list-final-separator    = { y },
    list-pair-separator     = { y },
    range-phrase            = { \translate{to (numerical range)} },
    qualifier-mode          = brackets,
    separate-uncertainty    = true,
    multi-part-units        = single,
    retain-explicit-plus    = true,
}

% Declaración de unidades propias
\DeclareSIUnit\torr{torr}           % Presión (tradicional)
\DeclareSIUnit\atm{atm}             % Presión
\DeclareSIUnit\molar{M}             % Concentración molar
\DeclareSIUnit\M{\molar}            % Concentración molar (alternativa)
\DeclareSIUnit\kcal{kcal}           % Energía
\DeclareSIUnit\cal{cal}             % Energía    
\DeclareSIUnit\mol{\mole}           % Cantidad de materia (en español)
\DeclareSIUnit\uma{u}               % Masa atómica
\DeclareSIUnit\h{\hour}             % Tiempo (abreviatura)
\DeclareSIUnit\hora{\hour}          % Tiempo (en español)
\DeclareSIUnit\min{\minute}         % Tiempo (abreviatura)  -- Podría dar problemas
\DeclareSIUnit\minuto{\minute}      % Tiempo (en español)
\DeclareSIUnit\Gm{\giga\m}          % Longitud
\DeclareSIUnit\Mm{\mega\m}          % Longitud
\DeclareSIUnit\hm{\hecto\m}         % Longitud
\DeclareSIUnit\cg{\centi\g}         % Longitud
\DeclareSIUnit\dam{\deca\m}         % Longitud
\DeclareSIUnit\dg{\deci\g}          % Masa
\DeclareSIUnit\hg{\hecto\g}         % Masa
% CHEMICAL PACKAGE %

% Cargar paquetes chemfig y chemformula
\usepackage{chemfig, chemformula}
    \setchemfig{atom sep=2em}
    \setchemformula{frac-style = nicefrac}

% modificación de nodos de chemfig para utilizar fira math
\renewcommand*\printatom[1]{\ensuremath{\mathsf{#1}}}
% XSIM PACKAGE %

% Carga xsim
\usepackage[no-files]{xsim}

% Carga el estilo que se va a usar
\loadxsimstyle{ged}

% Configuración general
\xsimsetup{
    % path                  = {xsim},
    exercise/template     = {gedmargin},
    exercise/name         = {},
    exercise/print        = {true},
    solution/template     = {gedsolution},
    solution/name         = {Autoevaluación: },
    solution/print        = {true},
    % exercise/within       = section,
    % exercise/the-counter  = \thesection.\arabic{exercise},
}

% xsim Exercise properties
%\DeclareExerciseProperty{source} % para indicar de dónde saqué el ejercicio
% TIKZ PACKAGE %

% Comandos para cargar tikz y pgfplots
\usepackage{tikz, pgfplots}

% Librerías adicionales que va a usar tikz
\usetikzlibrary{
  shapes.geometric,
  positioning,
  arrows.meta}

% Configuración general de pgfplots
\pgfplotsset{compat=1.9}


% Definición de estilos propios para pgfplots

\pgfplotsset{
  posicion tiempo/.style={
    xlabel={Tiempo [s]},
    ylabel={Posición [m]},
    width=4cm,
    axis lines=left,
    axis x line=middle,
    xtick distance=2,
    ytick distance=20,
    %minor tick num=4,
    grid=both,
    grid style={solid,lightgray},
    minor grid style={solid,very thin},
    font=\scriptsize,
    label style={font=\tiny}
}}

% Alternativa de configuración para expresar unidades como fracciones
% en la forma mol/L (en lugar de con exponentes negativos)
\sisetup{
    per-mode                  = symbol,
    per-symbol                = /,
    bracket-unit-denominator  = true,
}

\usepackage{icomma}
\usepackage[gen]{eurosym}
\usepackage{multicol}
\usepackage{tabularray}
\usepackage{wrapfig2}


% Configuración de la cabecera, en otro documento
% Header LUANCO (PACKAGE FANCYHDR) %

\usepackage{fancyhdr}
\pagestyle{fancy}

%\addtolength{\headwidth}{\marginparsep}
%\addtolength{\headwidth}{\marginparwidth}
%\renewcommand{\headwidth}{1.1\linewidth}
%\fancyheadoffset[LR]{0.1\linewidth}
\fancyhead{

  \begin{tblr}{
    % width=1.2\linewidth,
    colspec = {Q[c,h]lX[r]},
    stretch = 0,
    rowsep = 0pt,
    column{1} = {colsep=0pt},
    column{2} = {leftsep=10pt, font=\itshape},
    rows = {font=\footnotesize},
    % hlines = {3-4}{0.5pt},
    % vline{2-5} = {1pt},
  }
    \includegraphics[height=1.2cm]{D:/DOCENCIA/banco-de-ejercicios/.pandoc-config/assets/logo_ies_luanco.jpg} & {\emph{Dpto. de} \\ \emph{Física y Química}} & { {\normalsize  - } \\  - Curso 2024-25} \\
  \end{tblr}
}
\renewcommand{\headrulewidth}{0pt}

\fancyfoot[C]{\sffamily\fontsize{9pt}{9pt}\selectfont\thepage}

% END MODIFICATIONS


% Definitions
% DEFINICIÓN DE MACROS PROPIAS

% Entalpía estándar
\newcommand{\entalpia}[1][]{\Delta _{#1} H\textdegree}
\newcommand{\entalpiade}[2][]{\Delta _{#1} H\textdegree \left[\ch{#2}\right]}

% Entropía estándar
\newcommand{\entropia}[1][]{\Delta _{#1} S\textdegree}
\newcommand{\entropiade}[2][]{\Delta _{#1} S\textdegree \left[\ch{#2}\right]}

% Energía libre de Gibbs estándar
\newcommand{\Gibbs}[1][]{\Delta _{#1} G\textdegree}
\newcommand{\Gibbsde}[2][]{\Delta _{#1} G\textdegree \left[\ch{#2}\right]}

% Concentración
\newcommand{\conc}[2][]{[ \ch{#2} ]_{#1}}

% Presión parcial
\newcommand{\pparcial}[1]{p_{\ch{#1}}}

% Potencial de pila estándar / potencial de reducción estándar
\newcommand{\potE}[2]{E\textdegree (\ch{#1}/\ch{#2})}

% % END MODIFICATIONS

\begin{document}

\begin{exercise}Realiza los siguientes cambios de unidades usando
factores de conversión y escribiendo el resultado en notación
científica.

\begin{enumerate}
\def\labelenumi{\alph{enumi})}
\item
  \(\qty{8,9}{\dm} \cdot \frac{\qty{1}{\dam}}{\qty{100}{\dm}} = \frac{8,9 \cdot 1}{100} \unit{\dam} = \qty{8.9e-2}{\dam}\)
\item
  \(\qty{0,84}{\cm} \cdot \frac{\qty{e4}{\um}}{\qty{1}{\cm}} = \frac{0,84 \cdot 10^4}{1} \unit{\um}= \qty{8400}{\um} = \qty{8.4e3}{\um}\)
\item
  \(\qty{34}{\dg} \cdot \frac{\qty{1}{\kg}}{10^4 \unit{\dg}} = \frac{34 \cdot 1}{10^4} \unit{\kg} = \qty{0.0034}{\kg} = \qty{3.4e-3}{\kg}\)
\item
  \(\qty{8,9}{\Mm} \cdot \frac{10^5 \unit{\dam}}{\qty{1}{\Mm}} = \frac{8,9 \cdot 10^5}{1} = \qty{890000}{\dam} = \qty{8.9e5}{\dam}\)
\item
  \(76\text{ días} \cdot \frac{\qty{24}{\h}}{1\text{ día}} \frac{\qty{3600}{\s}}{\qty{1}{\h}} = \frac{76 \cdot 24 \cdot 3600}{1} \unit{\s}= \qty{6566400}{\s} = \qty{6.57e6}{\s}\)
\item
  \(\qty{5432}{\s} \cdot \frac{\qty{1}{\h}}{\qty{3600}{\s}} = \frac{5432 \cdot 1}{3600} \unit{\h} = \qty{1,509}{\h}\)
\item
  \(3\text{ años} \cdot \frac{365\text{ días}}{1\text{ año}} \frac{\qty{24}{\h}}{1\text{ día}} \frac{\qty{60}{\min}}{\qty{1}{\h}} = \frac{3 \cdot 365 \cdot 24 \cdot 60}{1} \unit{\min} = \qty{1576800}{\min} = \qty{1.577e6}{\min}\)
\item
  \(\qty{789}{\mm} \cdot \frac{10^6 \unit{\nm}}{\qty{1}{\Gm}} = \frac{789 \cdot 10^6}{1} \unit{\nm}= \qty{789000000}{\nm} = \qty{7.89e8}{\nm}\)
\item
  \(\qty{45}{\Gm} \cdot \frac{10^6 \unit{\km}}{\qty{1}{\Gm}} = \frac{45 \cdot 10^6}{1} \unit{\km} = \qty{45000000}{\km} = \qty{4.5e7}{\km}\)
\item
  \(\qty{9,1}{\ug} \cdot \frac{\qty{1}{\mg}}{10^3 \unit{\ug}} = \frac{9,1 \cdot 1}{10^3} \unit{\mg} = \qty{0.0091}{\mg} = \qty{9.1e-3}{\mg}\)
\end{enumerate}

\end{exercise}

\begin{exercise}Realiza los siguientes cambios de unidades usando
factores de conversión y escribiendo el resultado en notación
científica.

\begin{enumerate}
\def\labelenumi{\alph{enumi})}
\item
  \(\qty{71}{\m} \cdot \frac{\qty{100}{\cm}}{\qty{1}{\m}} = \frac{71 \cdot 100}{1} \unit{\cm} = \qty{7100}{\cm} = \qty{7.1e3}{\cm}\)
\item
  \(\qty{8,9}{\dam} \cdot \frac{\qty{100}{\dm}}{\qty{1}{\dam}} = \frac{8.9 \cdot 100}{1} \unit{\dm} = \qty{890}{\dm} = \qty{8.9e2}{\dm}\)
\item
  \(\qty{0,841}{\mm} \cdot \frac{10^3 \unit{\um}}{\qty{1}{\dm}} = \frac{0,841 \cdot 10^3}{1} \unit{\um} = \qty{841}{\um} = \qty{8.41e2}{\um}\)
\item
  \(\qty{78,9}{\dm} \cdot \frac{10^8 \unit{\nm}}{\qty{1}{\dm}} = \frac{78,9 \cdot 10^8}{1} \unit{\nm} = \qty{7890000000}{\nm} = \qty{7.89e9}{\nm}\)
\item
  \(\qty{34}{\min} \cdot \frac{\qty{60}{\s}}{\qty{1}{\min}} = \frac{34 \cdot 60}{1} \unit{\s} = \qty{2040}{\s} = \qty{2.04e3}{\s}\)
\item
  \(\qty{5}{\Mm} \cdot \frac{10^4 \unit{\hm}}{\qty{1}{\Mm}} = \frac{5 \cdot 10^4}{1} \unit{\hm} = \qty{50000}{\hm} = \qty{5e4}{\hm}\)
\item
  \(\qty{8,9}{\Gm} \cdot \frac{10^6 \unit{\km}}{\qty{1}{\Gm}} = \frac{8,9 \cdot 10^6}{1} \unit{\Gm} = \qty{8900000}{\Gm} = \qty{8.9e6}{\Gm}\)
\item
  \(\qty{8,7}{\hg} \cdot \frac{10^4 \unit{\cg}}{\qty{1}{\hg}} = \frac{8.7 \cdot 10^4}{1} \unit{\cg} = \qty{87000}{\cg} = \qty{8.7e4}{\cg}\)
\item
  \(\qty{49}{\kg} \cdot \frac{10^5 \unit{\cg}}{\qty{1}{\kg}} = \frac{49 \cdot 10^5}{1} \unit{\cg} = \qty{4900000}{\cg} = \qty{4.9e6}{\cg}\)
\item
  \(\qty{9,1}{\mg} \cdot \frac{10^3 \unit{\ug}}{\qty{1}{\mg}} = \frac{9,1 \cdot 10^3}{1} \unit{\ug} = \qty{9100}{\ug} = \qty{9.1e3}{\ug}\)
\end{enumerate}

\end{exercise}



\end{document}
