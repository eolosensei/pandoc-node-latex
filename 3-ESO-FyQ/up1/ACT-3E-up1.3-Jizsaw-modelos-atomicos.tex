% Options for packages loaded elsewhere
\PassOptionsToPackage{unicode}{hyperref}
\PassOptionsToPackage{hyphens}{url}
\PassOptionsToPackage{dvipsnames,svgnames*,x11names*}{xcolor}
%
\documentclass[
  spanish,
]{article}
\usepackage[sfdefault,lining]{FiraSans}
\usepackage{amsmath}
\usepackage{ifxetex,ifluatex}
\ifnum 0\ifxetex 1\fi\ifluatex 1\fi=0 % if pdftex
  \usepackage[T1]{fontenc}
  \usepackage[utf8]{inputenc}
  \usepackage{textcomp} % provide euro and other symbols
  \usepackage{amssymb}
\else % if luatex or xetex
  \usepackage{unicode-math}
  \defaultfontfeatures{Scale=MatchLowercase}
  \defaultfontfeatures[\rmfamily]{Ligatures=TeX,Scale=1}
\fi
% Use upquote if available, for straight quotes in verbatim environments
\IfFileExists{upquote.sty}{\usepackage{upquote}}{}
\IfFileExists{microtype.sty}{% use microtype if available
  \usepackage[]{microtype}
  \UseMicrotypeSet[protrusion]{basicmath} % disable protrusion for tt fonts
}{}
\makeatletter
\@ifundefined{KOMAClassName}{% if non-KOMA class
  \IfFileExists{parskip.sty}{%
    \usepackage{parskip}
  }{% else
    \setlength{\parindent}{0pt}
    \setlength{\parskip}{6pt plus 2pt minus 1pt}}
}{% if KOMA class
  \KOMAoptions{parskip=half}}
\makeatother
\usepackage{xcolor}
\IfFileExists{xurl.sty}{\usepackage{xurl}}{} % add URL line breaks if available
\IfFileExists{bookmark.sty}{\usepackage{bookmark}}{\usepackage{hyperref}}
\hypersetup{
  pdflang={es-ES},
  colorlinks=true,
  linkcolor=Maroon,
  filecolor=Maroon,
  citecolor=Blue,
  urlcolor=Blue,
  pdfcreator={LaTeX via pandoc}}
\urlstyle{same} % disable monospaced font for URLs
\usepackage[a4paper,headheight=10mm,headsep=10mm,top=30mm,right=25mm]{geometry}
\usepackage{longtable,booktabs}
\usepackage{calc} % for calculating minipage widths
% Correct order of tables after \paragraph or \subparagraph
\usepackage{etoolbox}
\makeatletter
\patchcmd\longtable{\par}{\if@noskipsec\mbox{}\fi\par}{}{}
\makeatother
% Allow footnotes in longtable head/foot
\IfFileExists{footnotehyper.sty}{\usepackage{footnotehyper}}{\usepackage{footnote}}
\makesavenoteenv{longtable}
\usepackage{graphicx}
\makeatletter
\def\maxwidth{\ifdim\Gin@nat@width>\linewidth\linewidth\else\Gin@nat@width\fi}
\def\maxheight{\ifdim\Gin@nat@height>\textheight\textheight\else\Gin@nat@height\fi}
\makeatother
% Scale images if necessary, so that they will not overflow the page
% margins by default, and it is still possible to overwrite the defaults
% using explicit options in \includegraphics[width, height, ...]{}
\setkeys{Gin}{width=\maxwidth,height=\maxheight,keepaspectratio}
% Set default figure placement to htbp
\makeatletter
\def\fps@figure{htbp}
\makeatother
\setlength{\emergencystretch}{3em} % prevent overfull lines
\providecommand{\tightlist}{%
  \setlength{\itemsep}{0pt}\setlength{\parskip}{0pt}}
\setcounter{secnumdepth}{-\maxdimen} % remove section numbering
\everymath{\displaystyle}
\ifxetex
  % Load polyglossia as late as possible: uses bidi with RTL langages (e.g. Hebrew, Arabic)
  \usepackage{polyglossia}
  \setmainlanguage[]{}
\else
  \usepackage[shorthands=off,main=spanish]{babel}
\fi
\ifluatex
  \usepackage{selnolig}  % disable illegal ligatures
\fi

\author{}
\date{}


% % MODIFICATIONS MADE BY ME

% MODIFICATIONS MADE BY ME

%\usepackage[sfdefault,lining]{FiraSans} %% option 'sfdefault' activates Fira Sans as the default text font
\usepackage[fakebold]{firamath-otf}
\renewcommand*\oldstylenums[1]{{\firaoldstyle #1}}
\setmathfont{Fira Math}


% SIUNITX PACKAGE %

% Cargar siunitx
\usepackage{siunitx}

% Configuración general
\sisetup{detect-all}

\sisetup{
    exponent-product        = \cdot,
    per-mode                = reciprocal,
    output-decimal-marker   = {,},
    group-digits            = integer,
    %text-celsius            = ^^b0\kern -\scriptspace C, % soluciona problemas con el símbolo de grados
    %math-celsius            = ^^b0\kern -\scriptspace C,
    list-final-separator    = { y },
    list-pair-separator     = { y },
    range-phrase            = { \translate{to (numerical range)} },
    qualifier-mode          = brackets,
    separate-uncertainty    = true,
    multi-part-units        = single,
    retain-explicit-plus    = true,
}

% Declaración de unidades propias
\DeclareSIUnit\torr{torr}           % Presión (tradicional)
\DeclareSIUnit\atm{atm}             % Presión
\DeclareSIUnit\molar{M}             % Concentración molar
\DeclareSIUnit\M{\molar}            % Concentración molar (alternativa)
\DeclareSIUnit\kcal{kcal}           % Energía
\DeclareSIUnit\cal{cal}             % Energía    
\DeclareSIUnit\mol{\mole}           % Cantidad de materia (en español)
\DeclareSIUnit\uma{u}               % Masa atómica
\DeclareSIUnit\h{\hour}             % Tiempo (abreviatura)
\DeclareSIUnit\hora{\hour}          % Tiempo (en español)
\DeclareSIUnit\min{\minute}         % Tiempo (abreviatura)  -- Podría dar problemas
\DeclareSIUnit\minuto{\minute}      % Tiempo (en español)
\DeclareSIUnit\Gm{\giga\m}          % Longitud
\DeclareSIUnit\Mm{\mega\m}          % Longitud
\DeclareSIUnit\hm{\hecto\m}         % Longitud
\DeclareSIUnit\cg{\centi\g}         % Longitud
\DeclareSIUnit\dam{\deca\m}         % Longitud
\DeclareSIUnit\dg{\deci\g}          % Masa
\DeclareSIUnit\hg{\hecto\g}         % Masa
% CHEMICAL PACKAGE %

% Cargar paquetes chemfig y chemformula
\usepackage{chemfig, chemformula}
    \setchemfig{atom sep=2em}
    \setchemformula{frac-style = nicefrac}

% modificación de nodos de chemfig para utilizar fira math
\renewcommand*\printatom[1]{\ensuremath{\mathsf{#1}}}
% XSIM PACKAGE %

% Carga xsim
\usepackage[no-files]{xsim}

% Carga el estilo que se va a usar
\loadxsimstyle{ged}

% Configuración general
\xsimsetup{
    % path                  = {xsim},
    exercise/template     = {gedmargin},
    exercise/name         = {},
    exercise/print        = {true},
    solution/template     = {gedsolution},
    solution/name         = {Autoevaluación: },
    solution/print        = {true},
    % exercise/within       = section,
    % exercise/the-counter  = \thesection.\arabic{exercise},
}

% xsim Exercise properties
%\DeclareExerciseProperty{source} % para indicar de dónde saqué el ejercicio
% TIKZ PACKAGE %

% Comandos para cargar tikz y pgfplots
\usepackage{tikz, pgfplots}

% Librerías adicionales que va a usar tikz
\usetikzlibrary{
  shapes.geometric,
  positioning,
  arrows.meta}

% Configuración general de pgfplots
\pgfplotsset{compat=1.9}


% Definición de estilos propios para pgfplots

\pgfplotsset{
  posicion tiempo/.style={
    xlabel={Tiempo [s]},
    ylabel={Posición [m]},
    width=4cm,
    axis lines=left,
    axis x line=middle,
    xtick distance=2,
    ytick distance=20,
    %minor tick num=4,
    grid=both,
    grid style={solid,lightgray},
    minor grid style={solid,very thin},
    font=\scriptsize,
    label style={font=\tiny}
}}

% Alternativa de configuración para expresar unidades como fracciones
% en la forma mol/L (en lugar de con exponentes negativos)
\sisetup{
    per-mode                  = symbol,
    per-symbol                = /,
    bracket-unit-denominator  = true,
}

\usepackage{icomma}
\usepackage[gen]{eurosym}
\usepackage{multicol}
\usepackage{tabularray}
\usepackage{wrapfig2}


% Configuración de la cabecera, en otro documento
% Header LUANCO (PACKAGE FANCYHDR) %

\usepackage{fancyhdr}
\pagestyle{fancy}

%\addtolength{\headwidth}{\marginparsep}
%\addtolength{\headwidth}{\marginparwidth}
%\renewcommand{\headwidth}{1.1\linewidth}
%\fancyheadoffset[LR]{0.1\linewidth}
\fancyhead{

  \begin{tblr}{
    % width=1.2\linewidth,
    colspec = {Q[c,h]lX[r]},
    stretch = 0,
    rowsep = 0pt,
    column{1} = {colsep=0pt},
    column{2} = {leftsep=10pt, font=\itshape},
    rows = {font=\footnotesize},
    % hlines = {3-4}{0.5pt},
    % vline{2-5} = {1pt},
  }
    \includegraphics[height=1.2cm]{D:/DOCENCIA/banco-de-ejercicios/.pandoc-config/assets/logo_ies_luanco.jpg} & {\emph{Dpto. de} \\ \emph{Física y Química}} & { {\normalsize  - } \\  - Curso 2024-25} \\
  \end{tblr}
}
\renewcommand{\headrulewidth}{0pt}

\fancyfoot[C]{\sffamily\fontsize{9pt}{9pt}\selectfont\thepage}

% END MODIFICATIONS


% Definitions
% DEFINICIÓN DE MACROS PROPIAS

% Entalpía estándar
\newcommand{\entalpia}[1][]{\Delta _{#1} H\textdegree}
\newcommand{\entalpiade}[2][]{\Delta _{#1} H\textdegree \left[\ch{#2}\right]}

% Entropía estándar
\newcommand{\entropia}[1][]{\Delta _{#1} S\textdegree}
\newcommand{\entropiade}[2][]{\Delta _{#1} S\textdegree \left[\ch{#2}\right]}

% Energía libre de Gibbs estándar
\newcommand{\Gibbs}[1][]{\Delta _{#1} G\textdegree}
\newcommand{\Gibbsde}[2][]{\Delta _{#1} G\textdegree \left[\ch{#2}\right]}

% Concentración
\newcommand{\conc}[2][]{[ \ch{#2} ]_{#1}}

% Presión parcial
\newcommand{\pparcial}[1]{p_{\ch{#1}}}

% Potencial de pila estándar / potencial de reducción estándar
\newcommand{\potE}[2]{E\textdegree (\ch{#1}/\ch{#2})}

% % END MODIFICATIONS

\begin{document}

\hypertarget{el-modelo-de-dalton}{%
\section{El modelo de Dalton}\label{el-modelo-de-dalton}}

A finales del siglo XVIII y comienzos del XIX, científicos como
Lavoisier, Proust y Dalton investigaron la conservación de la masa y la
relación entre las masas de las sustancias que intervienen en una
reacción química. Para explicar los resultados experimentales que
obtuvieron, en 1808 J. Dalton (1766-1844) publicó una serie de
enunciados, también llamados postulados, que completaban la que hoy
conocemos como teoría atómica de Dalton, que es la primera teoría
atómica basada en datos científicos.

Los postulados de Dalton se pueden resumir así:

\begin{enumerate}
\def\labelenumi{\arabic{enumi})}
\tightlist
\item
  La materia está formada por átomos indivisibles.
\item
  Cada elemento está formado por átomos iguales: tienen la misma masa y
  las mismas propiedades químicas.
\item
  Los átomos de distintos elementos tienen masas y propiedades químicas
  diferentes.
\item
  En las reacciones químicas, los átomos ni se crean ni se destruyen,
  solo cambian su distribución en las sustancias.
\item
  Los átomos de diferentes elementos se combinan para dar compuestos
  químicos, y lo hacen siempre en una proporción fija para cada tipo de
  compuesto posible.
\end{enumerate}

En sus postulados, Dalton establece que la materia no es continua, sino
de naturaleza corpuscular, y establece la diferencia entre elemento y
compuesto.

\hypertarget{el-modelo-atuxf3mico-de-thomson}{%
\section{El modelo atómico de
Thomson}\label{el-modelo-atuxf3mico-de-thomson}}

En 1904, J. J. Thomson idea un modelo atómico en el que el átomo es una
especie de esfera de carga positiva continua y esponjosa que contiene
casi toda la masa. Los electrones están incrustados en ella de forma
similar a como lo están las pasas en un bizcocho; por eso, se llamó el
``pudin de pasas''. Actualmente, sería más fácil hacer un símil con una
magdalena rellena de chips de chocolate.
\includegraphics[width=0.75\textwidth,height=\textheight]{FQ 3E 2015 SM Química-p76c.png}

Al incluir los electrones, este modelo explica los fenómenos de
electrización mediante la ganancia o pérdida de electrones: los átomos
pueden perder o ganar electrones, obteniendo una carga positiva (si
pierden electrones) o negativa (si ganan electrones).

\hypertarget{el-modelo-atuxf3mico-de-rutherford}{%
\section{El modelo atómico de
Rutherford}\label{el-modelo-atuxf3mico-de-rutherford}}

En 1909, E. Rutherford (1871-1937) y sus colaboradores, H. Geiger
(1882-1945) y E. Marsden (1889-1970), realizaron un experimento
revolucionario. Bombardearon una lámina de oro muy fina con partículas
cargadas positivamente que se desplazaban a gran velocidad.

\begin{figure}
\centering
\includegraphics[width=0.75\textwidth,height=\textheight]{FQ 3E OXF 2015 Química-p67a.png}
\caption{Esquema del experimento de Rutherford}
\end{figure}

Como se observa en el dibujo, y de acuerdo con el modelo atómico de
Thomson, lo que tendría que haber ocurrido es que las partículas
positivas hubieran atravesado la lámina sin ser apreciablemente
desviadas de su trayectoria rectilínea. Sin embargo, este fue el
resultado del experimento:

\begin{itemize}
\tightlist
\item
  La mayor parte de las partículas atravesaron la finísima lámina de oro
  sin cambiar la dirección (como era de esperar).
\item
  Algunas se desviaron considerablemente.
\item
  Sorprendentemente, algunas partículas rebotaron hacia la fuente de
  emisión.
\end{itemize}

Tras estos inesperados resultados, Rutherford llegó a las siguientes
conclusiones: - El hecho de que las partículas positivas que se dirigen
a gran velocidad hacia la lámina de oro la atraviesen sin desviarse
indica que el átomo es, en su mayor parte, espacio vacío. - El hecho de
que algunas partículas positivas procedentes de la fuente se desvíen
indica que han pasado cerca de una zona del átomo que también tiene
carga positiva y las ha repelido. - El hecho de que algunas partículas
positivas reboten hacia la fuente emisora indica que existen choques
directos contra una zona del átomo muy densa y fuertemente positiva, que
denominó núcleo atómico.

Rutherford pensaba que existía cierto parecido entre la estructura del
sistema solar y la del átomo, donde los electrones serían los planetas y
el núcleo, el Sol. Supuso que, igual que los planetas giran alrededor
del Sol, los electrones deberían hacerlo alrededor del núcleo. Por ello,
describió el átomo con dos zonas bien diferenciadas:

\begin{figure}
\centering
\includegraphics[width=0.75\textwidth,height=\textheight]{FQ 3E 2015 SM Química-p77c.png}
\caption{Modelo de Rutherford}
\end{figure}

\begin{itemize}
\tightlist
\item
  Una zona central del átomo muy pequeña, muy densa y cargada
  positivamente, pues es donde se encuentran los protones, a la que
  llamó \textbf{núcleo}.
\item
  Una zona periférica en la que los electrones, cargados negativamente,
  giran alrededor del núcleo y a cierta distancia del él, a la que llamó
  \textbf{corteza}.
\end{itemize}

De este modo, los resultados del experimento de Rutherford se explicaban
de una forma muy sencilla: las partículas positivas que pasaban cerca
del núcleo se desvían mucho de su trayectoria rectilínea, debido a la
repulsión entre cargas positivas. Aquellas partículas que chocaban
directamente contra el núcleo rebotan en la dirección de la que
proceden. Y las partículas que pasan alejadas del núcleo no se desvían.
Como la mayoría de las partículas del experimento no se desviaron,
Rutherford concluyó que la corteza y el núcleo estaban separados por un
enorme \textbf{espacio vacío} (enorme respecto al tamaño del núcleo).

\hypertarget{la-naturaleza-eluxe9ctrica-de-la-materia}{%
\section{La naturaleza eléctrica de la
materia}\label{la-naturaleza-eluxe9ctrica-de-la-materia}}

En el siglo VI a. C., Tales de Mileto había explicado que frotando un
paño de lana con un trozo de ámbar, ambos materiales son capaces de
atraer materiales muy ligeros, como plumas o cabellos. Este fenómeno se
denominó electricidad (\emph{elektron} significa «ámbar» en griego).
Posteriormente, Cisternay du Fay (1698-1739) y Benjamin Franklin
(1706-1790) describieron la existencia de dos tipos de cargas eléctricas
y estudiaron los fenómenos de electrización.

Los fenómenos de electrización y los relacionados con la corriente
eléctrica se investigaron a fondo a finales del siglo XIX, poniendo de
manifiesto que el átomo es divisible y que está formado por partículas
más pequeñas que tienen carga eléctrica.

A finales del siglo XIX y comienzos del XX, varios científicos
realizaron diferentes experimentos con tubos de vidrio, como el del
dibujo, que contenían un gas a baja presión al que se sometía a
descargas eléctricas de alto voltaje.

Estas experiencias permitieron a J. J. Thomson (1856-1940) identificar
la partícula responsable de la carga eléctrica negativa, el electrón, y
a E. Goldstein (1860-1930), la partícula responsable de la carga
eléctrica positiva, el protón. Con estos experimentos, ambos científicos
pudieron calcular el valor de la carga eléctrica y la masa de ambas
partículas.

Sin embargo, algo no encajaba: la suma de la masa de los protones más la
de los electrones era más pequeña que la masa del átomo en su conjunto.
Esto hizo a los plantearse que podía existir algo más: una partícula con
masa, pero sin carga eléctrica.

En 1932, J. Chadwick (1891-1974) bombardeó una lámina de berilio con
partículas positivas y observó que emitía una radiación de gran energía.
Posteriormente, demostró que esa radiación estaba formada por unas
partículas eléctricamente neutras, que denominó neutrones, cuya masa era
un poco mayor que la del protón. Con ello, todas las partículas
subatómicas estaban al descubierto:

\begin{longtable}[]{@{}
  >{\raggedright\arraybackslash}p{(\columnwidth - 6\tabcolsep) * \real{0.2133}}
  >{\centering\arraybackslash}p{(\columnwidth - 6\tabcolsep) * \real{0.2267}}
  >{\centering\arraybackslash}p{(\columnwidth - 6\tabcolsep) * \real{0.2400}}
  >{\centering\arraybackslash}p{(\columnwidth - 6\tabcolsep) * \real{0.3200}}@{}}
\toprule\noalign{}
\begin{minipage}[b]{\linewidth}\raggedright
Partícula
\end{minipage} & \begin{minipage}[b]{\linewidth}\centering
Carga
\end{minipage} & \begin{minipage}[b]{\linewidth}\centering
Masa
\end{minipage} & \begin{minipage}[b]{\linewidth}\centering
Descubrimiento
\end{minipage} \\
\midrule\noalign{}
\endhead
\bottomrule\noalign{}
\endlastfoot
Electrón & \qty{-1.602e-19}{\C} & \qty{9.109e-31}{\kg} & J. J. Thomson,
1897 \\
Protón & \qty{1.602e-19}{\C} & \qty{1.673e-27}{\kg} & E. Goldstein,
1886 \\
Neutrón & 0 & \qty{1.6749e-27}{\kg} & J. Chadwick, 1932 \\
\end{longtable}

La carga del electrón es la más pequeña que existe, y por eso recibe el
nombre de carga eléctrica elemental. Observa que, si no tenemos en
cuenta el signo, la carga del electrón es la misma que la del protón.
Por tanto, podemos afirmar que en un cuerpo eléctricamente neutro el
número de electrones debe ser igual al número de protones.

\hypertarget{los-nuevos-modelos-atuxf3micos}{%
\section{Los nuevos modelos
atómicos}\label{los-nuevos-modelos-atuxf3micos}}

Según el modelo del átomo nuclear o planetario, los electrones giran a
gran velocidad en torno al núcleo sin que sepamos a qué distancia de él
se encuentran. Sin embargo, es un hecho conocido que cualquier carga
eléctrica que gire debe emitir energía en forma de radiación. Si esto
sucediera, el electrón iría perdiendo energía y se acercaría cada vez
más al núcleo describiendo una trayectoria espiral, y acabaría cayendo
sobre él.

En 1913, Niels Bohr (1885-1962) modificó el modelo atómico de Rutherford
mediante los siguientes postulados:

\begin{enumerate}
\def\labelenumi{\arabic{enumi})}
\tightlist
\item
  El electrón solo se mueve en unas órbitas circulares, sin que exista
  emisión de energía. El electrón, dependiendo de la órbita en la que se
  encuentre, tiene una determinada energía, que es tanto mayor cuanto
  más alejada está la órbita del núcleo.
\item
  La emisión de energía solo se produce cuando un electrón salta de un
  nivel energético (órbita) de mayor energía a otro de menor energía.
\end{enumerate}

A partir de 1916 se desarrollaron otros modelos atómicos para tratar de
explicar las propiedades químicas de los elementos. Así, los científicos
llegaron a la conclusión de que los electrones están distribuidos en
niveles y subniveles de energía que admiten un número máximo de
electrones. La distribución por niveles de los electrones de un átomo de
un elemento recibe el nombre de configuración electrónica del elemento.
Los electrones situados en el último nivel energético de un átomo se
denominan electrones de valencia, y son los responsables de las
propiedades químicas de las sustancias, como veremos en la próxima
unidad.



\end{document}
