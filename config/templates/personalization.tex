% MODIFICATIONS MADE BY ME

%\usepackage[sfdefault,lining]{FiraSans} %% option 'sfdefault' activates Fira Sans as the default text font
\usepackage[fakebold]{firamath-otf}
\renewcommand*\oldstylenums[1]{{\firaoldstyle #1}}
\setmathfont{Fira Math}


\usepackage{siunitx}
\sisetup{detect-all}
\sisetup{
    exponent-product        = \cdot,
    per-mode                = reciprocal,
    output-decimal-marker   = {,},
    group-digits            = integer,
    %text-celsius            = ^^b0\kern -\scriptspace C, % soluciona problemas con el símbolo de grados
    %math-celsius            = ^^b0\kern -\scriptspace C,
    list-final-separator    = { y },
    list-pair-separator     = { y },
    range-phrase            = { \translate{to (numerical range)} },
    qualifier-mode          = brackets,
    separate-uncertainty    = true,
    multi-part-units        = single,
    retain-explicit-plus    = true,
}
\DeclareSIUnit\torr{torr}
\DeclareSIUnit\atm{atm}
\DeclareSIUnit\molar{M}
\DeclareSIUnit\M{\molar}
\DeclareSIUnit\kcal{kcal}
\DeclareSIUnit\cal{cal}
\DeclareSIUnit\mol{\mole}
\DeclareSIUnit\uma{u}
\DeclareSIUnit\h{\hour}
\DeclareSIUnit\hora{\hour}
\DeclareSIUnit\min{\minute}
\DeclareSIUnit\minuto{\minute}


% Alternativa de configuración para expresar unidades como fracciones
% en la forma mol/L (en lugar de con exponentes negativos)
\sisetup{
    per-mode                  = symbol,
    per-symbol                = /,
    bracket-unit-denominator  = true,
}
\usepackage{chemfig, chemformula}
  \setchemfig{atom sep=2em}
  \setchemformula{frac-style = nicefrac}
\usepackage[no-files]{xsim}
  \loadxsimstyle{ged}
  \xsimsetup{
    % path                  = {xsim},
    exercise/template     = {$if(xsim.exercise.template)$$xsim.exercise.template$$else$gedmargin$endif$},
    exercise/name         = {$if(xsim.exercise.name)$$xsim.exercise.name$$else$$endif$},
    exercise/print        = {$if(xsim.exercise.print)$$xsim.exercise.print$$else$false$endif$},
    solution/template     = {$if(xsim.solution.template)$$xsim.solution.template$$else$gedsolution$endif$},
    solution/name         = {$if(xsim.solution.name)$$xsim.solution.name$$else$Solución: $endif$},
    solution/print        = {$if(xsim.solution.print)$$xsim.solution.print$$else$false$endif$},
    % exercise/within       = section,
    % exercise/the-counter  = \thesection.\arabic{exercise},
  }
\usepackage{icomma}
\usepackage[gen]{eurosym}
\usepackage{multicol}
\usepackage{fancyhdr}
\usepackage{tabularray}
\pagestyle{fancy}

\addtolength{\headwidth}{\marginparsep}
\addtolength{\headwidth}{\marginparwidth}
\renewcommand{\headwidth}{1.1\linewidth}
\fancyheadoffset[LR]{0.1\linewidth}
\fancyhead{

  \begin{tblr}{
%    width=1.2\linewidth,
    colspec = {Q[c,h]l|X[c]|c|},
    stretch = 0,
    rowsep = 0pt,
    column{1} = {colsep=0pt},
    column{2} = {leftsep=0pt, font=\itshape},
    rows = {font=\footnotesize},
    hlines = {3-4}{0.5pt},
%    vline{2-5} = {1pt},
  }
    \includegraphics{$data-dir$/logo_ies_luanco.jpg} & {Dpto. de \\ \emph{Física y Química}} & { {\normalsize $encabezado.titulo$} \\ $encabezado.subtitulo$} & {$encabezado.nivel$ \\ Curso 2023-24} \\
  \end{tblr}
}
\renewcommand{\headrulewidth}{0pt}



% Tikz & pgfplots
\usepackage{pgfplots}
\pgfplotsset{compat=1.9}
\usetikzlibrary{shapes.geometric}
\pgfplotsset{posicion tiempo/.style={
  xlabel={Tiempo [s]},
  ylabel={Posición [m]},
  width=4cm,
  axis lines=left,
  axis x line=middle,
  xtick distance=2,
  ytick distance=20,
  %minor tick num=4,
  grid=both,
  grid style={solid,lightgray},
  minor grid style={solid,very thin},
  font=\scriptsize,
  label style={font=\tiny}
}}


% END MODIFICATIONS


% Definitions
\newcommand{\entalpia}[1][]{\Delta _{#1} H\textdegree}
\newcommand{\entalpiade}[2][]{\Delta _{#1} H\textdegree \left[\ch{#2}\right]}
\newcommand{\entropia}[1][]{\Delta _{#1} S\textdegree}
\newcommand{\entropiade}[2][]{\Delta _{#1} S\textdegree \left[\ch{#2}\right]}
\newcommand{\Gibbs}[1][]{\Delta _{#1} G\textdegree}
\newcommand{\Gibbsde}[2][]{\Delta _{#1} G\textdegree \left[\ch{#2}\right]}
\newcommand{\conc}[2][]{[ \ch{#2} ]_{#1}}
\newcommand{\pparcial}[1]{p_{\ch{#1}}}
\newcommand{\potE}[2]{E\textdegree (\ch{#1}/\ch{#2})}

% xsim Exercise properties
%\DeclareExerciseProperty{source} % para indicar de dónde saqué el ejercicio

% modificación de nodos de chemfig para utilizar fira math
\renewcommand*\printatom[1]{\ensuremath{\mathsf{#1}}}